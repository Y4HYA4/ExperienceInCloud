\chapter*{General Introduction}
\addcontentsline{toc}{chapter}{General Introduction}

The impact of cloud computing on industry and end users cannot be overstated: the ubiquitous presence of software that runs on cloud networks has transformed many aspects of daily life. Startups and businesses can reduce costs and expand their offerings by leveraging cloud computing rather than purchasing and managing their own hardware and software. Independent developers now have the ability to launch globally accessible apps and online services. Researchers can now share and analyze data on previously unimaginable scales. \textbf{Furthermore}, internet users can quickly access software and storage in order to create, share, and store digital media in quantities that far exceed the computing capacity of their personal devices.
\newline
Despite the growing popularity of cloud computing, many people are unaware of its specifics. What is the cloud, how does it work, and what are the advantages for businesses, developers, researchers, governments, healthcare practitioners, and students?.\newline
I had the opportunity to learn more about all of this as part of my internship. and I'll try to go into more detail in this report as best as i can.


\section{Define The Cloud }

Cloud computing is defined as follows by the National Institute of Standards and Technology (NIST), a non-regulatory agency of the United States Department of Commerce with the mission of advancing innovation:
\newline
a model for providing ubiquitous, convenient, on-demand network access to a shared pool of configurable computing resources (e.g., networks, servers, storage, applications, and services) that can be rapidly provisioned and released with minimal management effort or interaction from service providers

\subsection{Brief History}
Many aspects of cloud computing can be traced back to the 1950s, when universities and businesses rented out mainframe computer computation time. At the time, renting was one of the only ways to gain access to computing resources because computing technology was too large and expensive for individuals to own or manage. By the 1960s, computer scientists such as John McCarthy of Stanford University and J.C.R. Licklider of the United States Department of Defense Advanced Research Projects Agency (ARPA) were proposing ideas that foreshadowed some of the major features of cloud computing today, such as the concept of computing as a public utility and the possibility of a network of computers that would allow people to access data and programs from anywhere in the world.

\subsection{First Steps}
Cloud computing, on the other hand, did not become a mainstream reality or a popular term until the first decade of the twenty-first century. Cloud services such as Amazon's Elastic Compute (EC2) and Simple Storage Service (S3) in 2006, Heroku in 2007, Google Cloud Platform in 2008, Alibaba Cloud in 2009, Windows Azure (now Microsoft Azure) in 2010, IBM's SmartCloud in 2011, and DigitalOcean in 2011 were introduced during this decade. These services enabled existing businesses to reduce costs by migrating their in-house IT infrastructure to cloud-based resources, and they provided resources for independent developers and small developer teams to create and deploy apps. Cloud-based applications, also known as Software as a Service (SaaS). During this time period, also became popular. Unlike on-premise software, which users must physically install and maintain on their machines, SaaS increased application availability by allowing users to access them on demand from a variety of devices.

\subsection{The Added Value}
Some of these cloud-based applications, such as Google's productivity apps (Gmail, Drive, and Docs) and Microsoft 365 (a cloud-based version of the Microsoft Office Suite), were launched by the same companies that launched cloud infrastructure services, while others, such as Adobe Creative Cloud, were launched as cloud-based applications using cloud providers' services. Based on these cloud providers' novel opportunities, new SaaS products and businesses emerged, such as Netflix's streaming services in 2007, the music platform Spotify in 2008, the file-hosting service Dropbox in 2009, the video conferencing service Zoom in 2012, and the communication tool Slack in 2013. Cloud-based IT infrastructure and cloud-based applications are now popular among businesses and individual users alike, and their market share is expected to grow.


\subsection{Thoughts..}
Cloud is simply an umbrella term for any IT resource that a consumer accesses via the Internet. Instead of relying on local infrastructure, the end user outsources ready-made resources and accesses them online.
I was always interested in cloud computing and had been dedicated to it for over a year. I managed to obtain 15 cloud certifications one after the other with one goal in mind.\textbf{Vision}. \newline
As vendors provide more customizable options, I expect to see an increasing number of small and medium-sized businesses migrate to cloud options. And large corporations looking to cut costs will continue to shift data storage to the cloud.
